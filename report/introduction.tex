Computations performed by devices often leave observable side effects beyond the normal output. Side channel attacks focus on exploiting these side effects to extract sensitive information. 

Speculative execution is one of the design techniques that has facilitated the increase in processor speed over the last decade. This technique is used in almost every modern processor and it guesses the the likely future execution path and prematurely executes instructions on this path. Consider an example where program's control flow depends on uncached value present in the memory. Instead of idling, processor guesses the path and then executes the instructions on this path speculatively. Whenever the value is fetched from memory, processors checks if its guess was correct. If it was indeed correct, it continues on the same path, otherwise it discards the results of the speculative execution and resumes execution from the correct path.

\textbf{cite here} demonstrated the viability of speculative attacks on modern processors using a proof of concept code which contains both attacker and victim in a single process. Attacker and victim also accessed a shared array to make this speculative attack possible. We extended this code to make possible spectre attack in a multi-process environment where attacker and victim are in different processes and access shared memory. Then we also demonstrated this attack on non-shared memory by using prime and probe to retrieve the data brought in cache hierarchy by \textit{transient instructions}. Transient instructions are those instructions which are speculatively executed but then the effect of these instructions on the CPU is reverted back.