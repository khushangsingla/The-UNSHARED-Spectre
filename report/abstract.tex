\documentclass{main}
\begin{document}
In the ever-evolving landscape of cybersecurity, the emergence of sophisticated threats has compelled us to reassess and fortify the modern computing systems. Spectre attacks have garnered significant attention due to their ability to exploit vulnerabilities arising from speculative execution. Almost every contemporary microprocessor architecture is vulnerable to spectre attacks. Spectre attacks, a class of side-channel attacks, exploit the speculative execution feature present in almost all modern processors. By manipulating the processor's speculative execution pathways, malicious actors can leak sensitive information from the victim's memory. This paper delves into Spectre attacks on non-shared memory. Non-shared memory is typically considered secure against side-channel attacks. However, we demonstrate Spectre attack on non-shared memory to leak the victim's private information. We aim to shed light on the intricacies of the Spectre attack on non-shared memory and contribute to the ongoing effort of making systems secure against these speculative attacks.
\end{document}